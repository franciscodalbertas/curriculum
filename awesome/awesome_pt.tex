%!TEX TS-program = xelatex
%!TEX encoding = UTF-8 Unicode
% Awesome CV LaTeX Template for CV/Resume
%
% This template has been downloaded from:
% https://github.com/posquit0/Awesome-CV
%
% Author:
% Claud D. Park <posquit0.bj@gmail.com>
% http://www.posquit0.com
%
%
% Adapted to be an Rmarkdown template by Mitchell O'Hara-Wild
% 23 November 2018
%
% Template license:
% CC BY-SA 4.0 (https://creativecommons.org/licenses/by-sa/4.0/)
%
%-------------------------------------------------------------------------------
% CONFIGURATIONS
%-------------------------------------------------------------------------------
% A4 paper size by default, use 'letterpaper' for US letter
\documentclass[11pt,a4paper,]{awesome-cv}

% Configure page margins with geometry
\usepackage{geometry}
\geometry{left=1.4cm, top=.8cm, right=1.4cm, bottom=1.8cm, footskip=.5cm}


% Specify the location of the included fonts
\fontdir[fonts/]

% Color for highlights
% Awesome Colors: awesome-emerald, awesome-skyblue, awesome-red, awesome-pink, awesome-orange
%                 awesome-nephritis, awesome-concrete, awesome-darknight

\definecolor{awesome}{HTML}{414141}

% Colors for text
% Uncomment if you would like to specify your own color
% \definecolor{darktext}{HTML}{414141}
% \definecolor{text}{HTML}{333333}
% \definecolor{graytext}{HTML}{5D5D5D}
% \definecolor{lighttext}{HTML}{999999}

% Set false if you don't want to highlight section with awesome color
\setbool{acvSectionColorHighlight}{true}

% If you would like to change the social information separator from a pipe (|) to something else
\renewcommand{\acvHeaderSocialSep}{\quad\textbar\quad}

\def\endfirstpage{\newpage}

%-------------------------------------------------------------------------------
%	PERSONAL INFORMATION
%	Comment any of the lines below if they are not required
%-------------------------------------------------------------------------------
% Available options: circle|rectangle,edge/noedge,left/right

\name{Francisco}{d'Albertas Gomes de Carvalho}

\position{Pesquisador}
\address{International Institute for Sustainability (IIS). Estrada Dona
Castorina, 124 Horto, Rio de Janeiro, Brasil 22460-320}

\mobile{+55 11 99121 8851}
\email{\href{mailto:francisco.albertas@gmail.com}{\nolinkurl{francisco.albertas@gmail.com}}}
\homepage{franciscodalbertas.netlify.app}
\github{franciscodalbertas}
\twitter{FdAlbertas}

% \gitlab{gitlab-id}
% \stackoverflow{SO-id}{SO-name}
% \skype{skype-id}
% \reddit{reddit-id}


\usepackage{booktabs}

\providecommand{\tightlist}{%
	\setlength{\itemsep}{0pt}\setlength{\parskip}{0pt}}

%------------------------------------------------------------------------------



% Pandoc CSL macros
\newlength{\cslhangindent}
\setlength{\cslhangindent}{1.5em}
\newlength{\csllabelwidth}
\setlength{\csllabelwidth}{2em}
\newenvironment{CSLReferences}[3] % #1 hanging-ident, #2 entry spacing
 {% don't indent paragraphs
  \setlength{\parindent}{0pt}
  % turn on hanging indent if param 1 is 1
  \ifodd #1 \everypar{\setlength{\hangindent}{\cslhangindent}}\ignorespaces\fi
  % set entry spacing
  \ifnum #2 > 0
  \setlength{\parskip}{#2\baselineskip}
  \fi
 }%
 {}
\usepackage{calc}
\newcommand{\CSLBlock}[1]{#1\hfill\break}
\newcommand{\CSLLeftMargin}[1]{\parbox[t]{\csllabelwidth}{\honortitlestyle{#1}}}
\newcommand{\CSLRightInline}[1]{\parbox[t]{\linewidth - \csllabelwidth}{\honordatestyle{#1}}}
\newcommand{\CSLIndent}[1]{\hspace{\cslhangindent}#1}

\begin{document}

% Print the header with above personal informations
% Give optional argument to change alignment(C: center, L: left, R: right)
\makecvheader

% Print the footer with 3 arguments(<left>, <center>, <right>)
% Leave any of these blank if they are not needed
% 2019-02-14 Chris Umphlett - add flexibility to the document name in footer, rather than have it be static Curriculum Vitae


%-------------------------------------------------------------------------------
%	CV/RESUME CONTENT
%	Each section is imported separately, open each file in turn to modify content
%------------------------------------------------------------------------------



\hypertarget{sobre-mim}{%
\section{Sobre mim}\label{sobre-mim}}

Sou biólogo com mestrado e doutorado em Ecologia e possuo 10 anos de
experiência trabalhando com pesquisa aplicada, com foco em entender como
o padrão espacial de paisagens modificadas pelo ser humano afeta a
biodiversidade, a provisão de serviços ecossistêmicos e o clima. Meus
principais interesses estão relacionados ao planejamento de paisagens,
análise espacial e à interface entre ciência, políticas públicas e
avaliação de impacto.

\hypertarget{habilidades}{%
\section{Habilidades}\label{habilidades}}

\hypertarget{r---markdown---sig---anuxe1lise-de-dados---modelagem-serviuxe7os-ecossistuxeamicos---restaurauxe7uxe3o-ecoluxf3gica---escrita-e-comunicauxe7uxe3o---google-earth-engine---priorizauxe7uxe3o-espacial}{%
\subsection{R - markdown - SIG - análise de dados - modelagem serviços
ecossistêmicos - restauração ecológica - escrita e comunicação - Google
Earth Engine - priorização
espacial}\label{r---markdown---sig---anuxe1lise-de-dados---modelagem-serviuxe7os-ecossistuxeamicos---restaurauxe7uxe3o-ecoluxf3gica---escrita-e-comunicauxe7uxe3o---google-earth-engine---priorizauxe7uxe3o-espacial}}

\hypertarget{idiomas}{%
\section{Idiomas}\label{idiomas}}

\hypertarget{portugues-nativo---ingluxeas-fluente---francuxeas-avanuxe7ado---espanhol-intermediuxe1rio}{%
\subsection{Portugues (nativo) - Inglês (fluente) - Francês (avançado) -
Espanhol
(intermediário)}\label{portugues-nativo---ingluxeas-fluente---francuxeas-avanuxe7ado---espanhol-intermediuxe1rio}}

\hypertarget{educauxe7uxe3o}{%
\section{Educação}\label{educauxe7uxe3o}}

\begin{cventries}
    \cventry{Doutorado em Ecologia}{Universidade de São Paulo}{São Paulo, Brasil}{2017-2022}{}\vspace{-4.0mm}
    \cventry{Mestrado em Ecologia}{Universidade de São Paulo}{São Paulo, Brasil}{2013-2015}{}\vspace{-4.0mm}
    \cventry{Bacharelado em Ciências Biológicas}{Universidade de São Paulo}{São Paulo, Brasil}{2006-2011}{}\vspace{-4.0mm}
\end{cventries}

\hypertarget{experiuxeancia-profissional}{%
\section{Experiência profissional}\label{experiuxeancia-profissional}}

\hypertarget{trabalhos}{%
\subsection{Trabalhos}\label{trabalhos}}

\begin{cventries}
    \cventry{Pesquisador e cientista de dados no projeto Trade, Development and the Environment HUB}{International Institute for Sustainability-IIS}{Rio de Janeiro, Brazil}{2022}{}\vspace{-4.0mm}
    \cventry{Técnico de pesquisa - Bolsa de treinamento técnico da Fundação de Amparo a Pesquisa de São Paulo-FAPESP como parte  do projeto Biodiversity and Ecosystem Service Scenarios Network (ScenNet)}{Laboratório de Ecologia da Paisagem e Conservação-LEPaC(Departmento de Ecologia, Universidade de São Paulo) }{São Paulo, Brasil}{2015-2017}{}\vspace{-4.0mm}
    \cventry{Analista ambiental junior (estágio) - Programa de monitoramento de áreas protegidas}{Instituto Socioambiental - ISA}{São Paulo, Brasil}{2012-2013}{}\vspace{-4.0mm}
    \cventry{Analista ambiental junior (estágio) - Estudo e avaliação de impacto ambiental nos setores de mineração e petróleo}{Mineral Engenharia e Ambiente}{São Paulo, Brasil}{2011-2012}{}\vspace{-4.0mm}
\end{cventries}

\hypertarget{consultoria}{%
\subsection{Consultoria}\label{consultoria}}

\begin{cventries}
    \cventry{Consultoria - Avaliação da condição de linha de base dos serviços ecossistêmicos prioritários em bacias hidrográficas potencialmente ameaçadas por operações de mineração no estado de Minas Gerais}{Aquaflora provendo serviços para Vale S.A.}{São Paulo, Brasil}{2021}{}\vspace{-4.0mm}
    \cventry{Consultoria - Monitoramento de desmatamento e incêndios dentro de áreas protegidas da Amazônia e do Cerrado.}{World Wildlife Fund - WWF Brazil }{São Paulo, Brasil}{2020-2021}{}\vspace{-4.0mm}
    \cventry{Consultoria - Mapeamento ameaças às áreas protegidas federais e estaduais no Brasil}{Instituto Socioambiental - ISA }{São Paulo, Brasil}{2017-2018}{}\vspace{-4.0mm}
    \cventry{Consultoria - Planejamento de corredores ecológicos entre terras indígenas do Paraná}{Klabin s/a Papel e Celulose }{Paraná, Brasil}{2016}{}\vspace{-4.0mm}
    \cventry{Consultoria - Inventário florístico de São Miguel Paulista, São Paulo}{Bloomberg Philanthropies e Global Road Safety }{São Paulo, Brasil}{2016}{}\vspace{-4.0mm}
\end{cventries}

\hypertarget{ensino}{%
\subsection{Ensino}\label{ensino}}

\begin{cventries}
    \cventry{Professor - Co-organização da disciplina Landscape use, para estudantes de master in management}{Fundação Getulio Vargas}{Rio de Janeiro, Brazil}{2022}{}\vspace{-4.0mm}
\end{cventries}

\hypertarget{experiuxeancia-internacional}{%
\section{Experiência Internacional}\label{experiuxeancia-internacional}}

\begin{cventries}
    \cventry{Estudante visitante supervisiondo por Andrew Balmford}{Conservation Science Group, University of Cambridge }{Cambridge, United Kingdom}{2020-2021}{}\vspace{-4.0mm}
\end{cventries}

\hypertarget{publicauxe7uxf5es}{%
\section{Publicações}\label{publicauxe7uxf5es}}

\hypertarget{bibliography}{}
\leavevmode\vadjust pre{\hypertarget{ref-dalbertas_agricultural_2023}{}}%
\CSLLeftMargin{1. }%
\CSLRightInline{d'Albertas, F., Ruggiero, P., Pinto, L. F. G., Sparovek,
G., \& Metzger, J. P. (2023). Agricultural certification as a
complementary tool for environmental law compliance. \emph{Biological
Conservation}, \emph{277}, 109847.
\url{https://doi.org/10.1016/j.biocon.2022.109847}}

\leavevmode\vadjust pre{\hypertarget{ref-gonzalez-chaves_evidence_2023}{}}%
\CSLLeftMargin{2. }%
\CSLRightInline{González-Chaves, A. D., Carvalheiro, L. G., Piffer, P.
R., d'Albertas, F., Giannini, T. C., Viana, B. F., \& Metzger, J. P.
(2023). Evidence of time-lag in the provision of ecosystem services by
tropical regenerating forests to coffee yields. \emph{Environmental
Research Letters}, \emph{18}(2), 025002.
\url{https://doi.org/10.1088/1748-9326/acb161}}

\leavevmode\vadjust pre{\hypertarget{ref-pashkevich_nine_2022}{}}%
\CSLLeftMargin{3. }%
\CSLRightInline{Pashkevich, M. D., d'Albertas, F., Aryawan, A. A. K.,
Buchori, D., Caliman, J.-P., Chaves, A. D. G., Hidayat, P., Kreft, H.,
Naim, M., Razafimahatratra, A., Turner, E. C., Zemp, D. C., \& Luke, S.
H. (2022). Nine actions to successfully restore tropical agroecosystems.
\emph{Trends in Ecology \& Evolution}.
\url{https://doi.org/10.1016/j.tree.2022.07.007}}

\leavevmode\vadjust pre{\hypertarget{ref-dalbertas_private_2021}{}}%
\CSLLeftMargin{4. }%
\CSLRightInline{d'Albertas, F., González-Chaves, A., Borges-Matos, C.,
Zago de Almeida Paciello, V., Maron, M., \& Metzger, J. P. (2021).
Private reserves suffer from the same location biases of public
protected areas. \emph{Biological Conservation}, \emph{261}, 109283.
\url{https://doi.org/10.1016/j.biocon.2021.109283}}

\leavevmode\vadjust pre{\hypertarget{ref-dalbertas_lack_2018}{}}%
\CSLLeftMargin{5. }%
\CSLRightInline{d'Albertas, F., Costa, K., Romitelli, I., Barbosa, J.
M., Vieira, S. A., \& Metzger, J. P. (2018). Lack of evidence of edge
age and additive edge effects on carbon stocks in a tropical forest.
\emph{Forest Ecology and Management}, \emph{407}, 57--65.
\url{https://doi.org/10.1016/j.foreco.2017.09.042}}

\leavevmode\vadjust pre{\hypertarget{ref-acosta_gaps_2018}{}}%
\CSLLeftMargin{6. }%
\CSLRightInline{Acosta, A. L., d'Albertas, F., Leite, M. de S., Saraiva,
A. M., \& Metzger, J. P. W. (2018). Gaps and limitations in the use of
restoration scenarios: A review. \emph{Restoration Ecology},
\emph{26}(6), 1108--1119. \url{https://doi.org/10.1111/rec.12882}}

\leavevmode\vadjust pre{\hypertarget{ref-metzger_best_2017}{}}%
\CSLLeftMargin{7. }%
\CSLRightInline{Metzger, J. P., Esler, K., Krug, C., Arias, M., Tambosi,
L., Crouzeilles, R., Acosta, A. L., Brancalion, P. H., D'Albertas, F.,
Duarte, G. T., Garcia, L. C., Grytnes, J.-A., Hagen, D., Jardim, A. V.
F., Kamiyama, C., Latawiec, A. E., Rodrigues, R. R., Ruggiero, P. G.,
Sparovek, G., \ldots{} Joly, C. (2017). Best practice for the use of
scenarios for restoration planning. \emph{Current Opinion in
Environmental Sustainability}, \emph{29}, 14--25.
\url{https://doi.org/10.1016/j.cosust.2017.10.004}}


\label{LastPage}~
\end{document}
