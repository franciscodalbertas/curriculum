%!TEX TS-program = xelatex
%!TEX encoding = UTF-8 Unicode
% Awesome CV LaTeX Template for CV/Resume
%
% This template has been downloaded from:
% https://github.com/posquit0/Awesome-CV
%
% Author:
% Claud D. Park <posquit0.bj@gmail.com>
% http://www.posquit0.com
%
%
% Adapted to be an Rmarkdown template by Mitchell O'Hara-Wild
% 23 November 2018
%
% Template license:
% CC BY-SA 4.0 (https://creativecommons.org/licenses/by-sa/4.0/)
%
%-------------------------------------------------------------------------------
% CONFIGURATIONS
%-------------------------------------------------------------------------------
% A4 paper size by default, use 'letterpaper' for US letter
\documentclass[11pt,a4paper,]{awesome-cv}

% Configure page margins with geometry
\usepackage{geometry}
\geometry{left=1.4cm, top=.8cm, right=1.4cm, bottom=1.8cm, footskip=.5cm}


% Specify the location of the included fonts
\fontdir[fonts/]

% Color for highlights
% Awesome Colors: awesome-emerald, awesome-skyblue, awesome-red, awesome-pink, awesome-orange
%                 awesome-nephritis, awesome-concrete, awesome-darknight

\definecolor{awesome}{HTML}{414141}

% Colors for text
% Uncomment if you would like to specify your own color
% \definecolor{darktext}{HTML}{414141}
% \definecolor{text}{HTML}{333333}
% \definecolor{graytext}{HTML}{5D5D5D}
% \definecolor{lighttext}{HTML}{999999}

% Set false if you don't want to highlight section with awesome color
\setbool{acvSectionColorHighlight}{true}

% If you would like to change the social information separator from a pipe (|) to something else
\renewcommand{\acvHeaderSocialSep}{\quad\textbar\quad}

\def\endfirstpage{\newpage}

%-------------------------------------------------------------------------------
%	PERSONAL INFORMATION
%	Comment any of the lines below if they are not required
%-------------------------------------------------------------------------------
% Available options: circle|rectangle,edge/noedge,left/right

\name{Francisco}{d'Albertas Gomes de Carvalho}

\position{Research Associate}
\address{Conservation Research Institute (UCCRI), Department of
Zoology,University of Cambridge}

\email{\href{mailto:fd370@cam.ac.uk}{\nolinkurl{fd370@cam.ac.uk}}}
\homepage{fdalbertas.com}
\github{franciscodalbertas}
\linkedin{franciscodalbertas}
\twitter{FdAlbertas}

% \gitlab{gitlab-id}
% \stackoverflow{SO-id}{SO-name}
% \skype{skype-id}
% \reddit{reddit-id}


\usepackage{booktabs}

\providecommand{\tightlist}{%
	\setlength{\itemsep}{0pt}\setlength{\parskip}{0pt}}

%------------------------------------------------------------------------------



% Pandoc CSL macros
% definitions for citeproc citations
\NewDocumentCommand\citeproctext{}{}
\NewDocumentCommand\citeproc{mm}{%
  \begingroup\def\citeproctext{#2}\cite{#1}\endgroup}
\makeatletter
 % allow citations to break across lines
 \let\@cite@ofmt\@firstofone
 % avoid brackets around text for \cite:
 \def\@biblabel#1{}
 \def\@cite#1#2{{#1\if@tempswa , #2\fi}}
\makeatother
\newlength{\cslhangindent}
\setlength{\cslhangindent}{1.5em}
\newlength{\csllabelwidth}
\setlength{\csllabelwidth}{3em}
\newenvironment{CSLReferences}[2] % #1 hanging-indent, #2 entry-spacing
 {\begin{list}{}{%
  \setlength{\itemindent}{0pt}
  \setlength{\leftmargin}{0pt}
  \setlength{\parsep}{0pt}
  % turn on hanging indent if param 1 is 1
  \ifodd #1
   \setlength{\leftmargin}{\cslhangindent}
   \setlength{\itemindent}{-1\cslhangindent}
  \fi
  % set entry spacing
  \setlength{\itemsep}{#2\baselineskip}}}
 {\end{list}}
\usepackage{calc}
\newcommand{\CSLBlock}[1]{\hfill\break\parbox[t]{\linewidth}{\strut\ignorespaces#1\strut}}
\newcommand{\CSLLeftMargin}[1]{\parbox[t]{\csllabelwidth}{\strut#1\strut}}
\newcommand{\CSLRightInline}[1]{\parbox[t]{\linewidth - \csllabelwidth}{\strut#1\strut}}
\newcommand{\CSLIndent}[1]{\hspace{\cslhangindent}#1}

\begin{document}

% Print the header with above personal informations
% Give optional argument to change alignment(C: center, L: left, R: right)
\makecvheader

% Print the footer with 3 arguments(<left>, <center>, <right>)
% Leave any of these blank if they are not needed
% 2019-02-14 Chris Umphlett - add flexibility to the document name in footer, rather than have it be static Curriculum Vitae
\makecvfooter
  {June 2024}
    {Francisco d'Albertas Gomes de Carvalho~~~·~~~Curriculum Vitae}
  {\thepage~ of \pageref{LastPage}~}


%-------------------------------------------------------------------------------
%	CV/RESUME CONTENT
%	Each section is imported separately, open each file in turn to modify content
%------------------------------------------------------------------------------



\section{About me}\label{about-me}

I am a landscape ecologist with a master's and Ph.D.~in Ecology, and my
work centers on assessing the impacts of agriculture on biodiversity,
climate, and ecosystem service outcomes, as well on the value of
ecological restoration.

\section{Skills}\label{skills}

\subsection{R - markdown - GIS - data analysis - ecosystem service
modelling - ecological restoration - writing and communication - Google
Earth Engine - spatial
prioritization}\label{r---markdown---gis---data-analysis---ecosystem-service-modelling---ecological-restoration---writing-and-communication---google-earth-engine---spatial-prioritization}

\section{Languages}\label{languages}

\subsection{Portuguese (native) - English (fluent) - French (advanced) -
Spanish
(intermediate)}\label{portuguese-native---english-fluent---french-advanced---spanish-intermediate}

\section{Education}\label{education}

\begin{cventries}
    \cventry{PhD in Ecology}{Universidade de São Paulo}{São Paulo, Brasil}{2017-2022}{}\vspace{-4.0mm}
    \cventry{Master in Ecology}{Universidade de São Paulo}{São Paulo, Brasil}{2013-2015}{}\vspace{-4.0mm}
    \cventry{Biological Sciences degree}{Universidade de São Paulo}{São Paulo, Brasil}{2006-2011}{}\vspace{-4.0mm}
\end{cventries}

\section{Professional experience}\label{professional-experience}

\subsection{Jobs}\label{jobs}

\begin{cventries}
    \cventry{Research associate}{Conservation Research Institute,Department of Zoology,University of Cambridge}{Cambridge, UK}{2024}{}\vspace{-4.0mm}
    \cventry{Researcher and data scientist at the project Trade, Development and the Environment HUB}{International Institute for Sustainability-IIS}{Rio de Janeiro, Brazil}{2022-2024}{}\vspace{-4.0mm}
    \cventry{Researcher - Sao Paulo Research Foundation Training fellowship  as part of the Biodiversity and Ecosystem Service Scenarios Network (ScenNet)}{Laboratório de Ecologia da Paisagem e Conservação-LEPaC(Departmento de Ecologia, Universidade de São Paulo) }{São Paulo, Brasil}{2015-2017}{}\vspace{-4.0mm}
    \cventry{Junior environmental analyst (Internship) - Protected Areas Monitoring Programe}{Instituto Socioambiental - ISA}{São Paulo, Brasil}{2012-2013}{}\vspace{-4.0mm}
    \cventry{Junior environmental analyst (Internship) - Oil and mining environmental impact evaluation }{Mineral Engenharia e Ambiente}{São Paulo, Brasil}{2011-2012}{}\vspace{-4.0mm}
\end{cventries}

\subsection{Consultancy}\label{consultancy}

\begin{cventries}
    \cventry{Consultancy - Evaluating the baseline condition of priority ecosystem services at watersheds potentially threatened by mining operation in Minas Gerais ||State}{Aquaflora providing services to Vale S.A.}{São Paulo, Brazil}{2021}{}\vspace{-4.0mm}
    \cventry{Consultancy - Monitoring deforestation and fire within Amazonia and Cerrado protected areas}{World Wildlife Fund - WWF Brazil }{São Paulo, Brasil}{2020-2021}{}\vspace{-4.0mm}
    \cventry{Consultancy - Mapping threats to federal and state protected areas in Brazil}{Instituto Socioambiental - ISA }{São Paulo, Brasil}{2017-2018}{}\vspace{-4.0mm}
    \cventry{Consultancy - Planning ecological corridors between indigenous lands}{Klabin s/a Papel e Celulose }{Paraná, Brasil}{2016}{}\vspace{-4.0mm}
    \cventry{Consultancy - Floristic inventory of urban areas in Sao Paulo}{Bloomberg Philanthropies e Global Road Safety }{São Paulo, Brasil}{2016}{}\vspace{-4.0mm}
\end{cventries}

\subsection{Teaching}\label{teaching}

\begin{cventries}
    \cventry{Professor - Co-organization with Carlos Alberto Scaramuzza of a short term course entitled Landscape use, for master in management students}{Fundação Getulio Vargas}{Rio de Janeiro, Brazil}{2022}{}\vspace{-4.0mm}
\end{cventries}

\section{Research experience}\label{research-experience}

\begin{cventries}
    \cventry{Visiting researcher supervised by Professor Andrew Balmford}{Conservation Science Group, University of Cambridge }{Cambridge, United Kingdom}{2020-2021}{}\vspace{-4.0mm}
\end{cventries}

\section{Publications}\label{publications}

\phantomsection\label{refs-ff2c724deb9e38dfc8883b24c904ba59}
\begin{CSLReferences}{0}{0}
\bibitem[\citeproctext]{ref-cesar_de_oliveira_european_2024}
\CSLLeftMargin{1. }%
\CSLRightInline{Cesar de Oliveira, S. E. M., Nakagawa, L., Lopes, G. R.,
Visentin, J. C., Couto, M., Silva, D. E., d'Albertas, F., Pavani, B. F.,
Loyola, R., \& West, C. (2024). The european union and united kingdom's
deforestation-free supply chains regulations: Implications for brazil.
\emph{Ecological Economics}, \emph{217}, 108053.
\url{https://doi.org/10.1016/j.ecolecon.2023.108053}}

\bibitem[\citeproctext]{ref-cerullo_conflicts_2024}
\CSLLeftMargin{2. }%
\CSLRightInline{Cerullo, G., Worthington, T., Brancalion, P., Brandão,
J., d'Albertas, F., Eyres, A., Swinfield, T., Edwards, D., \& Balmford,
A. (2024). Conflicts and opportunities for commercial tree plantation
expansion and biodiversity restoration across brazil. \emph{Global
Change Biology}, \emph{30}(3), e17208.
\url{https://doi.org/10.1111/gcb.17208}}

\bibitem[\citeproctext]{ref-dalbertas_yield_2023}
\CSLLeftMargin{3. }%
\CSLRightInline{d'Albertas, F., Sparovek, G., Pinto, L.-F. G.,
Hohlenwerger, C., \& Metzger, J.-P. (2023). Yield increases mediated by
pollination and carbon payments can offset restoration costs in coffee
landscapes. \emph{One Earth}.
\url{https://doi.org/10.1016/j.oneear.2023.11.007}}

\bibitem[\citeproctext]{ref-berger_quantify_2023}
\CSLLeftMargin{4. }%
\CSLRightInline{Berger, I., Dicks, L. V., \& Gomes de Carvalho, F.
d'Albertas. (2023). Quantify wild areas that optimize agricultural
yields. \emph{Nature}, \emph{622}(7984), 697--697.
\url{https://doi.org/10.1038/d41586-023-03312-y}}

\bibitem[\citeproctext]{ref-dalbertas_agricultural_2023}
\CSLLeftMargin{5. }%
\CSLRightInline{d'Albertas, F., Ruggiero, P., Pinto, L. F. G., Sparovek,
G., \& Metzger, J. P. (2023). Agricultural certification as a
complementary tool for environmental law compliance. \emph{Biological
Conservation}, \emph{277}, 109847.
\url{https://doi.org/10.1016/j.biocon.2022.109847}}

\bibitem[\citeproctext]{ref-dalbertas_private_2021}
\CSLLeftMargin{6. }%
\CSLRightInline{d'Albertas, F., González-Chaves, A., Borges-Matos, C.,
Zago de Almeida Paciello, V., Maron, M., \& Metzger, J. P. (2021).
Private reserves suffer from the same location biases of public
protected areas. \emph{Biological Conservation}, \emph{261}, 109283.
\url{https://doi.org/10.1016/j.biocon.2021.109283}}

\bibitem[\citeproctext]{ref-dalbertas_lack_2018}
\CSLLeftMargin{7. }%
\CSLRightInline{d'Albertas, F., Costa, K., Romitelli, I., Barbosa, J.
M., Vieira, S. A., \& Metzger, J. P. (2018). Lack of evidence of edge
age and additive edge effects on carbon stocks in a tropical forest.
\emph{Forest Ecology and Management}, \emph{407}, 57--65.
\url{https://doi.org/10.1016/j.foreco.2017.09.042}}

\bibitem[\citeproctext]{ref-acosta_gaps_2018}
\CSLLeftMargin{8. }%
\CSLRightInline{Acosta, A. L., d'Albertas, F., Leite, M. de S., Saraiva,
A. M., \& Metzger, J. P. W. (2018). Gaps and limitations in the use of
restoration scenarios: A review. \emph{Restoration Ecology},
\emph{26}(6), 1108--1119. \url{https://doi.org/10.1111/rec.12882}}

\bibitem[\citeproctext]{ref-metzger_best_2017}
\CSLLeftMargin{9. }%
\CSLRightInline{Metzger, J. P., Esler, K., Krug, C., Arias, M., Tambosi,
L., Crouzeilles, R., Acosta, A. L., Brancalion, P. H., D'Albertas, F.,
Duarte, G. T., Garcia, L. C., Grytnes, J.-A., Hagen, D., Jardim, A. V.
F., Kamiyama, C., Latawiec, A. E., Rodrigues, R. R., Ruggiero, P. G.,
Sparovek, G., \ldots{} Joly, C. (2017). Best practice for the use of
scenarios for restoration planning. \emph{Current Opinion in
Environmental Sustainability}, \emph{29}, 14--25.
\url{https://doi.org/10.1016/j.cosust.2017.10.004}}

\end{CSLReferences}


\label{LastPage}~
\end{document}
